\documentclass[12pt,a4paper,draft]{article}
\usepackage[utf8]{inputenc}
\usepackage[english]{babel}
\usepackage{amsmath}
\usepackage{amsfonts}
\usepackage{amssymb}
\usepackage[left=2cm,right=2cm,top=2cm,bottom=2cm]{geometry}
\author{Adrian Bach}
\title{Assessing visual representation of uncertainty in conservation policy makers-directed documents.}

\begin{document}
\maketitle

\tableofcontents

\section{Context \& Purpose}

Workshop in NINA.
Preliminary title "Communicating uncertainty from resource management models".
Group 2: visual representation of uncertainty and users perception.
Assessing the way uncertainty about scientific measures and models results is currently represented in documents directed to conservation policy-makers.
Kinkeldey et al. (2013) reviewed papers about uncertainty representation in scientific papers, identifying five different levels of dichotomies.
\begin{itemize}
\item Explicit (uncertainty obtained from the data) / Implicit (displaying all the possibilities)
\item Extrinsic (new objects to represent uncertainty, \textit{e.g.} skewers, glyphs, intervals) / Intrinsic (altering the data visuals, \textit{e.g.} play on colors and shapes)
\item Visually integral (make sense only combined with the data) / Visually separable (still make sense when  separated from the data)
\item Coincident (represented in the same place as the data) / Adjacent (represented in another box than the data one's)
\item Static (inanimate figure) / Dynamic (interactive figure)
\end{itemize}
Used these categories to characterize uncertainty representation in figures from a significant number of policy documents.
Also assessed the prevalence of uncertainty representation, the organisation, the natural feature concerned, the type of figure, and year of production. 
(Currently 24) figures from documents produced by different organisations, such as European Environmental Council (EEA), Scottish National Heritage (SNH), Joint Nature Conservation Committee (JNCC), Australia's Convention on Biological Diversity (CBD).
In order to produce a barplot of the frequency of use of each combination of representation.
Beside giving a better idea of the current habits for uncertainty representation in conservation, it will also be interesting to compare it with the best practice that will come out of the paper, and target more accurately the areas of possible improvement.

\section{Methods}

Search for documents with an internet browser 

\end{document}